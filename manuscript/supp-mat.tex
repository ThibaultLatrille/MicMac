%TC:ignore
\documentclass{article}
\usepackage{caption}
\usepackage{xcolor, colortbl}
\definecolor{BLUELINK}{HTML}{0645AD}
\definecolor{DARKBLUELINK}{HTML}{0B0080}
\PassOptionsToPackage{hyphens}{url}
\usepackage[colorlinks=false]{hyperref}
\hypersetup{colorlinks,
    linkcolor=DARKBLUELINK,
    anchorcolor=DARKBLUELINK,
    citecolor=DARKBLUELINK,
    filecolor=DARKBLUELINK,
    menucolor=DARKBLUELINK,
    urlcolor=BLUELINK
}
\PassOptionsToPackage{unicode}{hyperref}
\PassOptionsToPackage{naturalnames}{hyperref}

\usepackage[backend=biber,eprint=false,isbn=false,url=false,intitle=true,style=nature,date=year]{biblatex}
\addbibresource{references.bib}

\usepackage[margin=50pt]{geometry}
\usepackage{amssymb,amsfonts,amsmath,amsthm,mathtools}
\usepackage{lmodern}
\usepackage{bm,bbold}
\usepackage{verbatim}
\usepackage{float}
\usepackage{listings, enumerate, enumitem}
\usepackage[export]{adjustbox}
\usepackage{tabu}
\usepackage{longtable}
\tabulinesep=0.6mm
\newcommand\cellwidth{\TX@col@width}
\usepackage{hhline}
\setlength{\arrayrulewidth}{1.2pt}
\usepackage{multicol,multirow,array}
\usepackage{etoolbox}
\AtBeginEnvironment{tabu}{\footnotesize}
\usepackage{booktabs}
\usepackage{graphicx}
\pdfinclusioncopyfonts=1

\renewcommand{\thetable}{S\arabic{table}}
\renewcommand{\thefigure}{S\arabic{figure}}

\newcommand{\defEqual}{\stackrel{\text{def}}{=}}
\newcommand{\Multiply}{\cdot}
\newcommand{\MultiplyMatrix}{\times}
\newcommand{\UniDimArray}[1]{\bm{#1}}
\newcommand{\BiDimArray}[1]{\bm{#1}}
\newcommand{\tr}{^{\intercal}}
\newcommand{\inv}{^{-1}}
\DeclareMathOperator{\E}{\mathbb{E}}
\DeclareMathOperator{\Var}{\text{Var}}
\DeclareMathOperator{\Cov}{\text{Cov}}
\newcommand{\der}{\mathrm{d}}
\newcommand{\e}{\text{e}}
\newcommand{\Ne}{N_{\text{e}}}
\newcommand{\dn}{d_N}
\newcommand{\ds}{d_S}
\newcommand{\dnds}{\dn / \ds}
\newcommand{\pn}{\pi_N}
\newcommand{\ps}{\pi_S}
\newcommand{\pnps}{\pn / \ps}
\newcommand{\proba}{\mathbb{P}}
\newcommand{\pfix}{\proba_{\text{fix}}}
\newcommand{\Spi}{i}
\newcommand{\Spj}{j}
\newcommand{\NbrSpecies}{n}
\newcommand{\Time}{t}
\newcommand{\Trait}{P}
\newcommand{\Heredity}{h^2}
\newcommand{\HereditySpi}{\Heredity_{\Spi}}
\newcommand{\MeanTrait}{\bar{\Trait_{\Time}}}
\newcommand{\VecTrait}{\UniDimArray{\bar{\Trait}}}
\newcommand{\Root}{0}
\newcommand{\RootTrait}{\widehat{\theta}}
\newcommand{\VarPhy}{\Var \left[\MeanTrait\right]}
\newcommand{\VecOne}{\UniDimArray{1}}
\newcommand{\Cr}{\BiDimArray{C}}
\newcommand{\MutationRate}{\mu}
\newcommand{\SubRate}{q}
\newcommand{\NbrLoci}{L}
\newcommand{\VarPhenotype}{V_{\Trait}}
\newcommand{\VarPhenotypeSpi}{V_{\Trait, \Spi}}
\newcommand{\VarGenetic}{V_{\mathrm{G}}}
\newcommand{\VarGeneticSpi}{V_{\mathrm{G}, \Spi}}
\newcommand{\VarEnv}{V_{\mathrm{E}}}
\newcommand{\MatrixGenetic}{\BiDimArray{G}}
\newcommand{\VarMutation}{V_{\mathrm{M}}}
\newcommand{\GenArchi}{\NbrLoci \Multiply \E \left[ a^2 \right]}
\newcommand{\RateMut}{\sigma^2_{\mathrm{M}}}
\newcommand{\RatePhy}{\sigma^2_{\mathrm{B}}}
\newcommand{\RatePop}{\sigma^2_{\mathrm{W}}}
\newcommand{\RatePopSpi}{\sigma^2_{\mathrm{W}, \Spi}}
\newcommand{\VecRatePop}{\UniDimArray{\RatePop}}
\newcommand{\EstRatePhy}{\widehat{\RatePhy}}
\newcommand{\EstRatePop}{\widehat{\RatePop}}
\newcommand{\NIx}{\RatePhy / \RatePop}
\newcommand{\EstNIx}{\EstRatePhy / \EstRatePop}
\newcommand{\NI}{\frac{\RatePhy}{\RatePop}}

\newcommand{\StdSelection}{\sigma}
\newcommand{\VarSelection}{\StdSelection^2}


\title{Detection of selection for a quantitative traits through the scaled mutational variance at the phylogenetic and population-genetic level}

\author{
    \large
    \textbf{T. {Latrille}$^{1}$, T. {Gaboriau}$^{1}$, N. {Salamin}$^{1}$}\\
    \normalsize
    $^{1}$Université de Lausanne, Lausanne, Switzerland\\
    \texttt{\href{mailto:thibault.latrille@ens-lyon.org}{thibault.latrille@ens-lyon.org}} \\
}


\date{}

\begin{document}
    \maketitle
    \part*{Supplementary materials}
    \tableofcontents
    \clearpage

    \section{Theoretical derivation}
    \subsection{Definitions}

    We denote $\Trait_{i}$ the quantitative trait $\Trait$ for individual $i$.

    \begin{align}
        \VarPhenotype = \sum_{i=1}^{\Ne} \Trait_{i}^2 - \left( \sum_{i=1}^{\Ne} \Trait_{i} \right)^2
    \end{align}

    \subsection{Mutation-selection-drift equilibrium}

    Selection, whether positive or negative, depletes standing variation in the population and reduces to genetic variance.
    Hence, at equilibrium between mutation, selection and drift, the observed genetic variance is lower than the theoretical neutral variance:
    \begin{align}
        \VarGenetic & < 2 \Ne \VarMutation \\
        \iff \VarGenetic & < 4 \Ne \NbrLoci \MutationRate \E \left[ a^2 \right]
    \end{align}

    Under the House of cards model of stabilizing selection\cite{kingman_simple_1978}, the trait's fitness is normally distributed around an optimal value ($\theta$) and with standard deviation $\StdSelection_S$.
    Moreover, the trait is encoded by a few loci with large effects and the genetic variance at equilibrium between mutation, selection and drift is:
    \begin{gather}
        \VarGenetic = 2 \NbrLoci \MutationRate \VarSelection_S \label{eq-mutsel-HC}.
    \end{gather}

    Alternatively, under the Gaussian model of stabilizing selection\cite{lande_natural_1976}, the trait's fitness is also normally distributed around an optimal value ($\theta$) and with standard deviation $\StdSelection_S$.
    However, the trait is encoded by a large number of loci traits with small effect and the genetic variance at equilibrium is:
    \begin{gather}
        \VarGenetic = \sqrt{2 \NbrLoci \VarMutation \VarSelection_S} \label{eq-mutsel-G}.
    \end{gather}

    The boundary between the two models is given by \textcite{turelli_heritable_1984} as:
    \begin{align}
        20 \MutationRate \VarSelection_S &< \E \left[ a^2 \right] \text{ for the House of card model}, \\
        20 \MutationRate \VarSelection_S &> \E \left[ a^2 \right] \text{ for the geometric model}.
    \end{align}

    \subsection{Selection at the phylogenetic scale}
    If the trait is under directional selection, but the strength and direction of selection varies randomly from one generation to the next (variance $\VarSelection_D$), then the trait still follows a Brownian process:
    \begin{align}
        \VarPhy &= \Var \left( \Trait_z \right), \\
        &= \left(\frac{\VarGenetic}{\Ne} + \VarGenetic \VarSelection_D \VarGenetic \right) t. \label{eq-selection-vary-direct}
    \end{align}

    If the trait is under stabilizing selection, but the strength and direction of the optimum varies randomly from one generation to the next (variance $\VarSelection_{\theta}$), then the trait still follows a Brownian process:
    \begin{align}
        \VarPhy &= \VarSelection_{\theta} t. \label{eq-selection-vary-stabi}
    \end{align}

    Altogether, traits that are found to be under a long term Brownian process are not necessarily neutral.

    \printbibliography

\end{document}